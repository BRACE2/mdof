\protect\hyperlink{main-content}{Skip to main content}

\hypertarget{pst-scroll-pixel-helper}{}

\emph{} Back to top

\emph{} {Ctrl+K}

{}

\href{../index.html}{}

mdof

Site Navigation

\begin{itemize}
\tightlist
\item
  \href{index.html}{Reference}
\item
  \href{../theory/index.html}{Theory}
\item
  \href{../examples/index.html}{Examples}
\end{itemize}

\begin{itemize}
\tightlist
\item
  \href{https://github.com/BRACE2/mdof}{{\emph{}} {GitHub}}
\end{itemize}

{}

Site Navigation

\begin{itemize}
\tightlist
\item
  \href{index.html}{Reference}
\item
  \href{../theory/index.html}{Theory}
\item
  \href{../examples/index.html}{Examples}
\end{itemize}

\begin{itemize}
\tightlist
\item
  \href{https://github.com/BRACE2/mdof}{{\emph{}} {GitHub}}
\end{itemize}

Section Navigation

\begin{itemize}
\tightlist
\item
  \href{index.html}{mdof modules and top-level functions}
\end{itemize}

\begin{itemize}
\tightlist
\item
  \href{mdof.markov.html}{mdof.markov module}
\item
  \protect\hyperlink{}{mdof.realize module}
\item
  \href{mdof.modal.html}{mdof.modal module}
\item
  \href{mdof.transform.html}{mdof.transform module}
\end{itemize}

\hypertarget{rtd-footer-container}{}

\hypertarget{main-content}{}
\begin{itemize}
\tightlist
\item
  \href{../index.html}{\emph{}}
\item
  \href{index.html}{Reference}
\item
  mdof.realize module
\end{itemize}

\hypertarget{searchbox}{}

\hypertarget{module-mdof.realize}{}
\protect\hypertarget{mdof-realize-module}{}{}

\hypertarget{mdof.realize-module}{%
\section{\texorpdfstring{mdof.realize
module\protect\hyperlink{module-mdof.realize}{\#}}{mdof.realize module\#}}\label{mdof.realize-module}}

\begin{description}
\item[{ {{mdof.realize.}}{{era}}{(}\emph{{{Y}}},
\emph{{{**}}{{options}}}{)}\protect\hyperlink{mdof.realize.era}{\#}}]
System realization from Markov parameters (discrete impulse response
data). Ho-Kalman / Eigensystem Realization Algorithm (ERA)
\protect\hyperlink{id3}{{{[}}1{{]}}}
\protect\hyperlink{id4}{{{[}}2{{]}}}.

\begin{description}
\item[Parameters{:}]
\begin{itemize}
\item
  \textbf{Y} (\emph{array}) -- Markov parameters. dimensions:
  {\textbackslash{}((p,q,nt)\textbackslash{})}, where
  {\textbackslash{}(p\textbackslash{})} = number of outputs,
  {\textbackslash{}(q\textbackslash{})} = number of inputs, and
  {\textbackslash{}(nt\textbackslash{})} = number of Markov parameters.
\item
  \textbf{horizon} (\emph{int,} \emph{optional}) -- number of block rows
  in Hankel matrix = order of observability matrix. default:
  {\textbackslash{}(min(150, (nt-1)/2)\textbackslash{})}
\item
  \textbf{nc} (\emph{int,} \emph{optional}) -- number of block columns
  in Hankel matrix = order of controllability matrix. default:
  {\textbackslash{}(min(150, max(nt-1-\textbackslash{})}
  \texttt{horizon}{\textbackslash{}(, (nt-1)/2))\textbackslash{})}
\item
  \textbf{order} (\emph{int,} \emph{optional}) -- model order. default:
  {\textbackslash{}(min(20,\textbackslash{})}
  \texttt{horizon}{\textbackslash{}(/2)\textbackslash{})}
\end{itemize}
\item[Returns{:}]
realization in the form of state space coefficients \texttt{(A,B,C,D)}
\item[Return type{:}]
tuple of arrays
\end{description}

\hypertarget{references}{}
\hypertarget{references}{%
\subsection{\texorpdfstring{References\protect\hyperlink{references}{\#}}{References\#}}\label{references}}

{{{[}}\protect\hyperlink{id1}{1}{{]}}}

Ho, Β. L., \& Kálmán, R. E. (1966). Effective construction of linear
state-variable models from input/output functions: Die Konstruktion von
linearen Modeilen in der Darstellung durch Zustandsvariable aus den
Beziehungen für Ein-und Ausgangsgrößen. at-Automatisierungstechnik,
14(1-12), 545-548. (\url{https://doi.org/10.1524/auto.1966.14.112.545})

{{{[}}\protect\hyperlink{id2}{2}{{]}}}

Juang, J. N., \& Pappa, R. S. (1985). An eigensystem realization
algorithm for modal parameter identification and model reduction.
Journal of guidance, control, and dynamics, 8(5), 620-627.
(\url{https://doi.org/10.2514/3.20031})
\end{description}

\begin{description}
\item[{ {{mdof.realize.}}{{era\_dc}}{(}\emph{{{Y}}},
\emph{{{**}}{{options}}}{)}\protect\hyperlink{mdof.realize.era_dc}{\#}}]
System realization from Markov parameters (discrete impulse response
data). Eigensystem Realization Algorithm with Data Correlations (ERA/DC)
\protect\hyperlink{id7}{{{[}}3{{]}}}.

\begin{description}
\item[Parameters{:}]
\begin{itemize}
\item
  \textbf{Y} (\emph{array}) -- Markov parameters. dimensions:
  {\textbackslash{}((p,q,nt)\textbackslash{})}, where
  {\textbackslash{}(p\textbackslash{})} = number of outputs,
  {\textbackslash{}(q\textbackslash{})} = number of inputs, and
  {\textbackslash{}(nt\textbackslash{})} = number of Markov parameters.
\item
  \textbf{horizon} (\emph{int,} \emph{optional}) -- number of block rows
  in Hankel matrix = order of observability matrix. default:
  {\textbackslash{}(min(150, (nt-1)/2)\textbackslash{})}
\item
  \textbf{nc} (\emph{int,} \emph{optional}) -- number of block columns
  in Hankel matrix = order of controllability matrix. default:
  {\textbackslash{}(min(150, max(nt-1-\textbackslash{})}
  \texttt{horizon}{\textbackslash{}(, (nt-1)/2))\textbackslash{})}
\item
  \textbf{order} (\emph{int,} \emph{optional}) -- model order. default:
  {\textbackslash{}(min(20,\textbackslash{})}
  \texttt{horizon}{\textbackslash{}(/2)\textbackslash{})}
\item
  \textbf{a} (\emph{int,} \emph{optional}) --
  {\textbackslash{}((\textbackslash{}alpha)\textbackslash{})} number of
  block rows in Hankel of correlation matrix. default: 0
\item
  \textbf{b} (\emph{int,} \emph{optional}) --
  {\textbackslash{}((\textbackslash{}beta)\textbackslash{})} number of
  block columns in Hankel of correlation matrix. default: 0
\item
  \textbf{l} (\emph{int,} \emph{optional}) -- initial lag for data
  correlations. default: 0
\item
  \textbf{g} (\emph{int,} \emph{optional}) -- lags (gap) between
  correlation matrices. default: 1
\end{itemize}
\item[Returns{:}]
realization in the form of state space coefficients \texttt{(A,B,C,D)}
\item[Return type{:}]
tuple of arrays
\end{description}

\hypertarget{id6}{}
\hypertarget{references-1}{%
\subsection{\texorpdfstring{References\protect\hyperlink{id6}{\#}}{References\#}}\label{references-1}}

{{{[}}\protect\hyperlink{id5}{3}{{]}}}

Juang, J. N., Cooper, J. E., \& Wright, J. R. (1987). An eigensystem
realization algorithm using data correlations (ERA/DC) for modal
parameter identification.
(\url{https://ntrs.nasa.gov/citations/19870035963})
\end{description}

\begin{description}
\item[{ {{mdof.realize.}}{{srim}}{(}\emph{{{inputs}}},
\emph{{{outputs}}},
\emph{{{**}}{{options}}}{)}\protect\hyperlink{mdof.realize.srim}{\#}}]
System realization from input and output data, with output error
minimization method. System Realization Using Information Matrix (SRIM)
\protect\hyperlink{id10}{{{[}}4{{]}}}.

\begin{description}
\item[Parameters{:}]
\begin{itemize}
\item
  \textbf{inputs} (\emph{array}) -- input time history. dimensions:
  {\textbackslash{}((q,nt)\textbackslash{})}, where
  {\textbackslash{}(q\textbackslash{})} = number of inputs, and
  {\textbackslash{}(nt\textbackslash{})} = number of timesteps
\item
  \textbf{outputs} (\emph{array}) -- output response history.
  dimensions: {\textbackslash{}((p,nt)\textbackslash{})}, where
  {\textbackslash{}(p\textbackslash{})} = number of outputs, and
  {\textbackslash{}(nt\textbackslash{})} = number of timesteps
\item
  \textbf{horizon} (\emph{int,} \emph{optional}) -- number of steps used
  for identification (prediction horizon). default:
  {\textbackslash{}(min(300, nt)\textbackslash{})}
\item
  \textbf{order} (\emph{int,} \emph{optional}) -- model order. default:
  {\textbackslash{}(min(20,\textbackslash{})}
  \texttt{horizon}{\textbackslash{}(/2)\textbackslash{})}
\item
  \textbf{full} (\emph{bool,} \emph{optional}) -- if True, full SVD.
  default: True
\item
  \textbf{find} (\emph{string,} \emph{optional}) -- ``ABCD'' or ``AC''.
  default: ``ABCD''
\item
  \textbf{threads} (\emph{int,} \emph{optional}) -- number of threads
  used during the output error minimization method. default: 6
\item
  \textbf{chunk} (\emph{int,} \emph{optional}) -- chunk size in output
  error minimization method. default: 200
\end{itemize}
\item[Returns{:}]
realization in the form of state space coefficients \texttt{(A,B,C,D)}
\item[Return type{:}]
tuple of arrays
\end{description}

\hypertarget{id9}{}
\hypertarget{references-2}{%
\subsection{\texorpdfstring{References\protect\hyperlink{id9}{\#}}{References\#}}\label{references-2}}

{{{[}}\protect\hyperlink{id8}{4}{{]}}}

Juang, J. N. (1997). System realization using information matrix.
Journal of Guidance, Control, and Dynamics, 20(3), 492-500.
(\url{https://doi.org/10.2514/2.4068})
\end{description}

\href{mdof.markov.html}{\emph{}}

previous

mdof.markov module

\href{mdof.modal.html}{}

next

mdof.modal module

\emph{}

\emph{} On this page

\begin{itemize}
\tightlist
\item
  \protect\hyperlink{mdof.realize.era}{\texttt{era()}}
\item
  \protect\hyperlink{mdof.realize.era_dc}{\texttt{era\_dc()}}
\item
  \protect\hyperlink{mdof.realize.srim}{\texttt{srim()}}
\end{itemize}

© Copyright 2023, Chrystal Chern.\\

Created using \href{https://www.sphinx-doc.org/}{Sphinx} 7.1.2.\\

Built with the
\href{https://pydata-sphinx-theme.readthedocs.io/en/stable/index.html}{PyData
Sphinx Theme} 0.14.1.
